\chapter{Conclusion}
\label{chap:conclusion}
\section{Result}
The result is a prototype capable of handling the Java and C++, but lacks a good indication of software quality and complexity. It visualizes data-structures, but contains duplicate data. The product owner requested a system that could be used to identify good or bad libraries, something the system is incapable of, but provides a good bedrock for future development, graduation projects or master thesis.

\section{Future Work}
\label{sec:future}
The project could be continued in a master thesis as a research or personal project where some of the following items could be integrated or improved upon:

\begin{itemize}
    \item Integrate AI to do language recognition.
    \item Integrate virtual reality for visualization.
    \item Extend a REST \gls{api} to be more independent from systems \gls{gui}, but also fetch desired parts of parsed code through REST.
    \item Spacing as sizing algorithm could be implemented in a proper manner insuring children don't escape their parents nor would crash with siblings.
    \item File tree structure could be implemented and visualized.
    \item Front-end could be refactored to a properly segmented and professional system.
    \item The camera controls could be improved to become more intuitive
    \item Adding user settings/account capabilities for remembering locale, submitted repositories, custom styles and saving the world state properties to continue the inspection from the same point or preparation.
\end{itemize}

\section{Evaluation of group work}
Throughout the development, all team members worked well and tried their best to create the best product possible. Despite problematic tasks and new technologies to learn, the group progressed and overcame most hinders.

\section{Ending}
All in all the group is happy about the work done in this project and the knowledge acquired throughout. The group is happy about the quality and consider the project well made, although a few parts could have been done better.