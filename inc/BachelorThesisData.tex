\thesistitle{3D Representation of Complex Data Structures}
\thesisshorttitle{Codebase Visualizer 3D} % use this if you have a very long title and want something shorter on the header pages
\thesisauthor{Butt, Zohaib}
\thesisauthorA{Holt, Kent Wincent}
\thesisauthorB{Hauge Torkelsen, Eldar}
%\thesisauthorC{}
\thesissupervisor{\supervisor{}}
%\thesissupervisorA{john smith} %second supervisor


% There use to be a number associated with projects, this would help identify which project was selected.  If you are told to add a project number then this line adds the number.
%\oppgaveNo{29E}

\nmtkeywords{WebGl, Antlr, visualization, programming language, parsing, 3D, Git, MIT}

\nmtdesc{Code-bases can quickly become very large and complex and it's usually difficult to get an overview of the code and it's complexity. We received a task form \productowner{} that would consume a Git repository link and will visualize the code-base that resides within the repository using different colors and shapes. Therefore we've produced a solution that tries to visualize the code in a 3D environment, in a clear and concise way. This resulted in a Web-App containing a back-end API written in Go and an Antlr-based language parser in Java, as well as a front-end written in HTML/CSS/JS. We have tried to follow a professional development method by the use of Scrum, Jira, Confluence as well as high code quality and documentation of both the source code and the development process it-self.}

\nmtoppdragsgiver{\NTNU}
\nmtcontact{\productowner{}}




\thesisdate{\ntnubachelorthesisdate}
\useyear{20.05.2019}

\nmtappnumber{} %number of appendixes
\nmtpagecount{} %currently auto calculated but might be wrong




\thesistitleNOR{3D visualisering av komplekse datastrukturer.}
\nmtkeywordsNOR{WebGl, Antlr, visualisering, programmeringspråk analyse, 3D, Git, MIT}
\nmtdescNOR{Kode-baser kan fort bli store og komplekse, samt kan det ta veldig lang tid å få et overblikk av koden og dens kompleksitet. Vi fikk i oppgave av \productowner{} å implementere et system som kunne ta imot en Git URL og visualisere kode-basen som finnes i Git repositoriet, ved bruk av ulike farger og modeller. Dermed har vi laget en løsning som prøver å visualisere kode i et 3D miljø på en oversiktlig og klar måte. Denne løsningen resulterte til en Web-App med en back-end API i Go og en Antlr-basert språk leser i Java, samt en front-end i HTML/CSS/JS. Vi har prøvd å holde til en professionell utviklings metode, ved bruk av Scrum, Jira og Confluence, samt høy kode kvalitet og dokumentasjon av kildekoden og prosessen.}

