\thesistitle{3D Representation of Complex Data Structures}
\thesisshorttitle{Codebase Visualizer 3D} % use this if you have a very long title and want something shorter on the header pages
\thesisauthor{Butt, Zohaib}
\thesisauthorA{Holt, Kent Wincent}
\thesisauthorB{Hauge Torkelsen, Eldar}
%\thesisauthorC{}
\thesissupervisor{\supervisor{}}
%\thesissupervisorA{john smith} %second supervisor


% There use to be a number associated with projects, this would help identify which project was selected.  If you are told to add a project number then this line adds the number.
%\oppgaveNo{29E}

\nmtkeywords{WebGl, Antlr, visualization, programming language, parsing, 3D, Git, MIT}

\nmtdesc{Codebases can quickly become large and complex. This makes it difficult to get an overview of the code. We received a task from \productowner{} to consume a Git URL and visualize the codebase that resides within Git repository, by using different colors and models. Therefore we've produced a solution that visualizes the code in a 3D environment in a clear and concise way. The solution is a Web-App containing a front-end written in HTML/SASS/JS and back-end API written in Go with an Antlr-based language parser in Java. We followed a professional development method by the use of Scrum, Jira and Confluence as well as high code quality and documentation of both the source code and the development process.}

%Kodebaser kan fort bli store og komplekse. Dette gjør at det kan ta veldig lang tid å få et overblikk over koden. Vi fikk i oppgave av \productowner{} å implementere et system som kunne ta imot en Git URL og visualisere kodebasen som finnes i Git repositoriet, ved bruk av ulike farger og modeller. Dermed har vi laget en løsning som visualiserer kode i et 3D miljø på en oversiktlig og klar måte. Løsningen er en Web-App med en front-end i HTML/SASS/JS og en back-end API i Go med en Antlr-basert språk leser i Java. Vi har holdt en professionell utviklings metode, ved bruk av Scrum, Jira og Confluence, samt høy kode kvalitet og dokumentasjon av kildekoden og prosessen.

\nmtoppdragsgiver{\NTNU}
\nmtcontact{\productowner{}}




\thesisdate{\ntnubachelorthesisdate}
\useyear{20.05.2019}

\nmtappnumber{} %number of appendixes
\nmtpagecount{} %currently auto calculated but might be wrong




\thesistitleNOR{3D Visualisering av Komplekse Datastrukturer}
\nmtkeywordsNOR{WebGl, Antlr, visualisering, programmeringspråk analyse, 3D, Git, MIT}
\nmtdescNOR{Kodebaser kan fort bli store og komplekse. Dette gjør at det kan ta veldig lang tid å få et overblikk over koden. Vi fikk i oppgave av \productowner{} å implementere et system som kunne ta imot en Git URL og visualisere kodebasen som finnes i Git repositoriet, ved bruk av ulike farger og modeller. Dermed har vi laget en løsning som visualiserer kode i et 3D miljø på en oversiktlig og klar måte. Løsningen er en Web-App med en front-end i HTML/SASS/JS og en back-end API i Go med en Antlr-basert språk leser i Java. Vi har holdt en professionell utviklings metode, ved bruk av Scrum, Jira og Confluence, samt høy kode kvalitet og dokumentasjon av kildekoden og prosessen.}

