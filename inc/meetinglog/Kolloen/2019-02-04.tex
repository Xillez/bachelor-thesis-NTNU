\section*{Meeting with Supervisor}

\subsection*{Participants}

\textbf{Convener}: \supervisor{}\\
\textbf{Facilitator}: \facilitator{}  \\
\textbf{Recorder}: \scrummaster{}  \\
\textbf{Others}: \groupleader{}                          \\
\textbf{Absent}: 

\subsection*{Agenda}
\begin{itemize}
    \item Feedback on report draft

\end{itemize}

\subsection*{Minutes}
Raporten ble godkjent og virker bra.
Viktigste er at dette  er et verktøy for dere.

Fristen for innlevering er nå riktig i studentweb, dere vet når dere skal levere.


Det som er vikti er at dere noterer alle avgjørelser underveis. Det er viktig for slutt raporten. Ja, det kan være nok med sprint review meetings for dette, i og med at dere har ukentlige sprint. Dette er den største feilen folk gjør. 


De delene av prosjektet synes er vanskelig, begynn med det. Det som er trivielt kan dere ta senere. 

Vi tenker å bruke Threejs er dette bra?
svar: Vet ikke, gjør ikke så mye med den delen av utvikling. Bare husk å skrive hvorfer dere tok dette valget. hvis det viser seg å ha svakheter er det da enklere å forklare til slutt.

Bruk produktet dere lager til å sjekke alternativer for Three.js for demonstrasjon.

\newpage