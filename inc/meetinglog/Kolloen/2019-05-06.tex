\section*{Meeting with Supervisor}

\subsection*{Participants}

\textbf{Convener}: \supervisor{}        \\
\textbf{Facilitator}: \facilitator{}    \\
\textbf{Recorder}: \scrummaster{}       \\
\textbf{Others}  \groupleader{}         \\
\textbf{Absent}: 

\subsection*{Agenda}
\begin{itemize}
    \item Hvordan referere til programm beskrivelse?
    \item Hvordan referere til ntnu-template?
    \item Hvordan kan vi oppdatere overleaf cache? 
    \item Hvor skal vi skrive om bruken av module pattern?
    \item Hvordan minimere kodesnutter?
    \item Max lengde av kodesnutt?
    \item Oppset av neste møte?
\end{itemize}

\subsection*{Minutes}
\subsubsection{Hvordan referere til program beskrivelse?}
De er ikke arkivert på web, så det kan være vanskelig. Gjør så godt dere kan, bruk beskrivelse på nett. Author er emneansvarlig etterfuøgt av institutt. 

\subsubsection{Hvordan referere til ntnu-template?}
Normal gruppe referering mot github. Authors er man contributers, ikke ha med publisher.

\subsubsection{Hvordan kan vi oppdatere overleaf cache?}
Vet ikke, kan snakke med noen av lærerene som har mer med overleaf.

\subsubsection{Hvor skal vi skrive om bruken av module pattern?}
Module pattern er mest i implementasjon. Kan referere fra design delen ned til implementasjon.

\subsubsection{Hvordan minimere kodesnutter?}
Bruk ... med // comment, for å forkaler hva som blir gjort i delen.

\subsubsection{Max lengde av kodesnutt?}
Blir kansje litt mye når deg går over 20 linjer. Kan være det blir lengre noen ganger.

\subsubsection{Oppset av neste møte}
Onsdag kl 10.15. 

\newpage