\chapter{Requirements}
\label{chap:requirements}

The group have divided the requirements in three sections result goals, optional requirements and learning outcomes. Result goals defines what the group expects to deliver as the end product. Operational requirements describe what the team expect from the system after a set amount of time and are listed to give an indication of the intended quality of the final system. Learning outcomes describe what the team wants to learn during development and thesis writing.

\section{Result Goal}
    \begin{itemize}
        \item Program must, at a minimum, be able to abstract and visualize language version up to Java 11.0.2 and C++17.
        \item Code visualization must be in 3D.
        \item Program must give an indication of the quality of the software code.  
        \item Program must be able to take a link to a git repository to use as basis for visualization.
        \item Program must be able to show potential execution path or program flow through the program.
        \item Program must have a public facing API for developers.
        \item Program must be dockerized. 
        \item The user must be able to interact with the 3D visualization to show implementation of data structures.
        \item The system must present complexity metrics to the user.
        \item The system must give an indication on where complexity arises.
    \end{itemize}
    
\section{Operational requirements}
    %%%%%
    % Eldar:    I think we need more information to do anything on effect goals, as what is there now is not easily quantifiable or verifiable. 
    %           It is to general to give any valuable information and is not related to what the effect of the program in action.
    %%%%%
    \begin{enumerate}  
        \item Quantitative goals for the system: 
        \begin{itemize}
            \item The system must handle 30 concurrent users with repository of less than 10 KLOC while using less than 10 minutes of processing time.
            \item The system must be able to handle a peak of at least 60 users.
            \item The system must be able to handle repositories of less than 100 KLOC. 
        \end{itemize}
        \item Qualitative goals for the system:
        \begin{itemize}
            \item The system must help developer(s) gain overview and an understanding of the code base.
            \item The system must let the user easily navigate through the visualization.
            \item The system must be naturally intuitive and easy-to-use for new users.
            \item The system must be accessible to people with minor disabilities like colorblindness. 
        \end{itemize}
    \end{enumerate}
        
\section{Learning outcome}
    \begin{itemize}
        \item Learn in depths WebGL technology.
        \item Learn professionalism in web development.
        \item Understand and follow DevOps methodology through out the project.
        \item Learn in depths dockerizing containers.
        \item Learn Front-end web frameworks.
        \item Learn about different code quality metrics for static analysis.
        \item Learn about professional web architectures.
        \item Learn about Infrastructure as Code.
    \end{itemize}

\section{Use Case}
\newglossaryentry{usecase}{name=use case, description={use case are usually a list of events, often used to describe the interaction between actors and system}}

To achieve more detailed view of the actual systems functionality that need to be implemented, it is important to visualize the system from users perspective without focusing on implementation details. To do this, \gls{usecase} and use-case diagram are used as they show users needs.



\begin{figure}[H]
\begin{tikzpicture}
\begin{umlsystem}{System}
\umlusecase[x=-2, y=0, name=highlightConn]{Highlight connection between data-structures}
\umlusecase[x=0, y=1, name=submitRepo]{Submit git repository}
\umlusecase[x=0, y=8.5, name=loadRepo]{Load other repository stored in DB}
\umlusecase[x=2, y=2, name=navigate]{Navigation of visualization}
\umlusecase[x=0, y=4, name=getImpl]{Get implementation of data-structure}
\umlusecase[x=0, y=5, name=qualityMet]{Select quality metrics} 
\umlusecase[x=0, y=6, name=lexer]{Adding a lexer file}
\umlusecase[x=2, y=3, name=color-scheme]{Add a custom color-scheme}
\umlusecase[x=0, y=7, name=GUI]{Change GUI layout}
\umlusecase[x=-3, y=10, name=codebase]{Get representation of codebase}
\end{umlsystem}

\umlactor[x=-9, y=4]{User}


\umlassoc{User}{highlightConn}
\umlassoc{User}{submitRepo}
\umlassoc{User}{loadRepo}
\umlassoc{User}{navigate}
\umlassoc{User}{getImpl}
\umlassoc{User}{qualityMet}
\umlassoc{User}{lexer}
\umlassoc{User}{color-scheme}
\umlassoc{User}{GUI}
\umlassoc{User}{codebase}

\umlinclude{codebase}{loadRepo}

\end{tikzpicture}
\caption{Use case diagram.}
\end{figure}


\subsection{High level use case descriptions}
\noindent\fbox{
    \parbox{\textwidth}{
        \textbf{Use case name:} Change GUI layout                                  \\
        \textbf{Actors:} User                                                        \\
        \textbf{Target:} Target is to change the GUI layout in the application.       \\
        \textbf{Description:} User can change the GUI layout by dragging and scaling windows. User can also add or remove a window of same type or any other type.   
    }
} \\ \\

\noindent\fbox{
    \parbox{\textwidth}{
        \textbf{Use case name:} Add a color-scheme                                 \\
        \textbf{Actors:} User                                                        \\
        \textbf{Target:} Target is to change the color-scheme used in the application. \\
        \textbf{Description:} User can change the color-scheme of application by specifying a set of colors from a palette on the home page.   
    }
} \\ \\


\noindent\fbox{
    \parbox{\textwidth}{
        \textbf{Use case name:} Add a lexer file                                  \\
        \textbf{Actors:} User                                                      \\
        \textbf{Target:} User can add new lexer file to the system.         \\
        \textbf{Description:} User can submit a new lexer files on the home page by clicking on the "upload lexer file" and choosing the file from a remote machine.
    }
} \\ \\

\noindent\fbox{
    \parbox{\textwidth}{
        \textbf{Use case name:} Select quality metrics                              \\
        \textbf{Actors:} User                                                       \\
        \textbf{Target:} User can select a specific quality metrics.         \\
        \textbf{Description:} User can choose between different quality metrics such as "connascence measure", "cyclomatic complexity" and "Halstead complexity measures" on the quality metrics window. The chosen metrics will then be used for analysis of the submitted codebase.
    }
} \\ \\

\noindent\fbox{
    \parbox{\textwidth}{
        \textbf{Use case name:} Submit git repository                               \\
        \textbf{Actors:} User                                                         \\
        \textbf{Target:} User can submit the git repository               \\
        \textbf{Description:} User can submit the git repository by adding repository link on the text field with name "Git Repository" and click submit to store repository in the database. The field can be found at home page.
    }
} \\ \\

\noindent\fbox{
    \parbox{\textwidth}{
        \textbf{Use case name:} Load other repository stored in DB                   \\
        \textbf{Actors:} User                                                         \\
        \textbf{Target:} User can load the repositories submitted by other users.     \\
        \textbf{Description:} Database has multiple repositories stored by different users. These repositories will be shown in the window named "Repositories". A user can click on available repositories to visualize it.
    }
} \\ \\

\noindent\fbox{
    \parbox{\textwidth}{
        \textbf{Use case name:} Get implementation of data-structure                   \\
        \textbf{Actors:} User                                                         \\
        \textbf{Target:} User can see the implementation of chosen data-structure    \\
        \textbf{Description:} This functionality is required to have at least one entry in the database so that the system can get the representation of codebase. Once the visualization of codebase is shown, user can choose between visualized data-structures to see the implementation on a window named "Implementation".
    }
} \\ \\

\noindent\fbox{
    \parbox{\textwidth}{
        \textbf{Use case name:} Navigation of visualization                   \\
        \textbf{Actors:} User                                                         \\
        \textbf{Target:} User can navigate in the 3D environment to see visualized models. \\
        \textbf{Description:} This functionality requires a submitted repository so that the representation of codebase can be shown. The navigation includes rotation, translation and zoom. The user can navigate in 3D environment using mouse or keyboard.
    }
} \\ \\

\noindent\fbox{
    \parbox{\textwidth}{
        \textbf{Use case name:} Highlight connection between data-structure          \\
        \textbf{Actors:} User                                                         \\
        \textbf{Target:} User can submit the git repository               \\
        \textbf{Description:} This functionality requires a submitted repository so that the representation of codebase can be shown. Once the codebase is visualized, the user can choose between different data-structures visualized to highlight it's links with other data-structures. Highlighting connections will brighten the color of all the links and connected data-structures, but also dull the color of all the unconnected data-structures.
    }
} \\ \\

\noindent\fbox{
    \parbox{\textwidth}{
        \textbf{Use case name:} Get representation of codebase       \\
        \textbf{Actors:} User             \\
        \textbf{Target:} Visualizing the codebase in 3D environment  \\
        \textbf{Description:} The user starts at the homepage where the git repository to be visualize needs to be submitted. Once the requested git link is submitted and stored in database, user is taken to loading page where user can see the status of processing codebase file from back-end server. After the processing is done user is taken to 3D visualization page where the visualization will show the different data-structures represented as a shape and a color. 
    }
} \\ \\
\section{Product backlog}
\begin{table}
    \resizebox{\columnwidth}{!}{
        \csvautotabular{inc/csv/finalIssuelist.csv}
    }
    \caption{Initial story estimates}
    \label{storyEstimate}
    
\end{table}

The initial story estimates in Table \ref{storyEstimate} resulted in 112 story points. The allocated time of 1200 workhours results in 10 hours per story point. The estimates will be re-estimated on the beginning of relevant sprint and the stories will change. 

\section{User stories}
%\begin{table}
%    \resizebox{\columnwidth}{!}{
%        \csvautotabular{csv/issueEstimate.csv}
%    }
%    \caption{Initial story estimates}
%    \label{storyEstimate}
%    
%\end{table}
%
%The initial story estimates in Table \ref{storyEstimate} resulted in 112 story points. The allocated time of 1200 workhours results in 10 hours per story point. The estimates will be re-estimated on the beginning of relevant sprint and the stories will change. 

\section{Domain model}

\begin{figure}[H]
\begin{tikzpicture}

\umlemptyclass[x=0, y=-3]{User}
\umlemptyclass[x=0, y=-6]{Visualization}
\umlclass[x=0, y=-10]{Quality metrics}{
    name : str\\
    value[] : int\\
    quality rank : str\\
}{}

\umlclass[x=5, y=-3]{Git Project}{
    URI : str \\
    commitLog[] : str
}{}
\umlemptyclass[x=5, y=-6]{Project Code}

\umlclass[x=10, y=-6]{Source File}{
    language : str \\
    content : str 
}{}
\umlclass[x=10, y=-12]{Language Parser}{
    language : str 
}{}

\umlclass[x=6.5, y=-8]{Syntax Rule}{
     syntax: str \\
}{}

\umlclass[x=5, y=-12]{Code Structure}{
    name : str\\
}{}

\umlemptyclass[x=0.5, y=-15]{Namespace}
\umlemptyclass[x=3, y=-15]{Class}
\umlclass[x=5, y=-15]{Function}{
}{}
\umlclass[x=7, y=-15]{Variable}{
    value : str\\
    name : str\\
    type : str\\
}{}
\umlemptyclass[x=9, y=-15]{Template}
\umlemptyclass[x=11, y=-15]{Call}

\umlclass[x=7, y=-18]{Argument}{
}{}

\umlassoc[geometry=-|, arg1=1..1, anchor2=-180, arg2=1..*] 
    {User}{Git Project}
\umlassoc[geometry=-|,  arg1=1..1, anchor2=-180, arg2=1..1 ]
    {Visualization}{Project Code}
\umlassoc[geometry=-|-, arg1=1..*, pos1=1, anchor2=-160, arg2=1..1, pos2=2.5]
    {Quality metrics}{Project Code}

\umlassoc[arg1=1..1, pos1=0.05, anchor2=-90, arg2=1..1, pos2=0.65] 
    {Source File}{Language Parser}
\umlassoc[geometry=-|, arg1=1..1, pos1=1.55, anchor2=90, arg2=1..1, pos2=1.9] 
    {Git Project}{Project Code}
\umlassoc[geometry=-|, arg1=1..1, pos1=0.1, anchor2=90, arg2=1..*, pos2=0.55] 
    {Project Code}{Source File}

\umluniassoc[arg1=0..*, angle1=10, angle2=50, loopsize=1.5cm]
    {Syntax Rule}{Syntax Rule}
    
\umlNarynode[x=6.5,y=-10, name=languageStructureSyntax, below]{}
    \umlassoc[arg1=0..*]{Syntax Rule}{languageStructureSyntax}  
    \umlassoc[geometry=|-, anchor1=120, arg1=1..1]{Language Parser}{languageStructureSyntax}  
    \umlassoc[geometry=|-, arg1=0..*]{Code Structure}{languageStructureSyntax} 

\umluniassoc[arg1=0..*, angle1=10, angle2=50, loopsize=1.5cm]
    {Code Structure}{Code Structure}


\umlinherit[geometry=|-|]{Namespace}{Code Structure}
\umlinherit[geometry=|-|]{Class}    {Code Structure}
\umlinherit[geometry=|-|]{Function} {Code Structure}
\umlinherit[geometry=|-|]{Variable} {Code Structure}
\umlinherit[geometry=|-|]{Template} {Code Structure}
\umlinherit[geometry=|-|]{Call}     {Code Structure}

\umlcompo[]{Variable}{Argument}
\umlassoc[geometry=-|, arg1=0..*, arg2=0..1, pos2=1.8]{Argument}{Call} 


\end{tikzpicture}
\caption{Domain model.}
\label{fig:domainModel}
\end{figure}

\newglossaryentry{recursion}{name=recursion, description={A computer programming technique involving the use of a procedure, subroutine, function, or algorithm that calls itself one or more times until a specified condition is met at which time the rest of each repetition is processed from the last one called to the first }}
\newglossaryentry{recursive}{name=recursive, description={Of, relating to, or involving \gls{recursion}}}
\newglossaryentry{syntax}{name=syntax, description={The way in which linguistic elements (such as words) are put together to form constituents (such as phrases or clauses)}}

Figure \ref{fig:domainModel} shows a crude overview of the subject area relating to the project. The code structure is language dependent and \gls{recursive} based on the \gls{syntax} rules relating to the structure. The relationship between a call, variable and Parameter relates to how a function-call can have parameters that define which overloaded version of a function is being called. This relationship is given as an example but there are many other similar relationships that have not been added for clarity, like how variables like classes can define the scope a function is being called from. This is a relationship between functions being defined as part of a class, variables being instances of said class and calls being called on said variable.

\Gls{syntax} rules are also \gls{recursive} as they can be defined by a set of rules or a character sequence.

\section{Risk assessment}

The initial task description and product owner didn't give any specific security requirements and therefore the group has done a risk assessment to identify any shortcomings that might happen and identify how to deal with them This risk assessment will be a part of the base for the security requirements.

\subsection{Identification and project risk analysis}

The table \ref{riskOverview} contains possible risks based on \cite{DBLP:journals/corr/abs-1708-02174} and compared with the system requirements. The probabilities and effects are estimates done by the core team. The priority is based on the minimal Manhattan distance from top right corner of table \ref{riskForm}.

c\begin{table}[H]
\centering
\resizebox{\columnwidth}{!}{
 \begin{tabular}{| c | c | l | c | c | c |} 
 \hline
 \cellcolor[HTML]{AFAFAF} Id & \cellcolor[HTML]{AFAFAF} Risk Type & \cellcolor[HTML]{AFAFAF} \cellcolor[HTML]{AFAFAF} Possible Risk & \cellcolor[HTML]{AFAFAF} Priority & \cellcolor[HTML]{AFAFAF} Probability & \cellcolor[HTML]{AFAFAF} Effect                     \\  
 \hline\hline
 1 & Organization & Losing important data & 7 & Low & Significant                           \\ 
 \hline
 2 & Technology & Technology components aren't scalable & 4 & Medium & Severe              \\
 \hline
 3 & Technology & Sanitation of input from external sources & 10 & Medium & Significant    \\
 \hline
 4 & Technology & Application instability & 5 & Low & Severe                                           \\
 \hline
 5 & Organization & Team members are unavailable for longer period of time & 6 & Medium & Significant            \\
 \hline
 6 & Requirement & Discovering problems in requirements at delivery & 1 & High & Severe     \\
 \hline
 7 & Estimate & Project size is under estimated & 3 & High & Significant                    \\
 \hline
 8 & Organization & Learning curves lead to delays & 2 & Very High & Moderate \\
 \hline
 9 & Technology & Integration testing environments aren't available & 9 & Low & Minor       \\
 \hline
 10 & Organization & Fail to follow software methodology & 8 & Medium & Minimal             \\
 \hline
 11 & Technology & Exposure of sensitive personal or corporate data & 11 & Very Low & Severe             \\
 \hline
 12 & Technology & Incapability of identifying individual or corporate entity & 12 & Very Low & Moderate             \\
 \hline
 13 & Technology & Illegal use of stored software & 13 & Low & Significant             \\
 \hline
\end{tabular}
}
\caption{Overview of risks}
\label{riskOverview}
\end{table}

\begin{table}[H]
    \centering
        \begin{tabular}{| c | c | c | c | c | c |} 
            \hline
            \cellcolor[HTML]{AFAFAF} & Minimal & Minor & 
                                       Moderate & Significant & 
                                       Severe    \\  
            \hline\hline
            Very High  &    \cellcolor[HTML]{FFEA00} & \cellcolor[HTML]{FFEA00} & 
                            \cellcolor[HTML]{FF0004} 8 & \cellcolor[HTML]{FF0004} &
                            \cellcolor[HTML]{FF0004}  \\ 
            \hline
            High       &    \cellcolor[HTML]{FFEA00} & \cellcolor[HTML]{FFEA00} & 
                            \cellcolor[HTML]{FFEA00} & \cellcolor[HTML]{FF0004} 7 & 
                            \cellcolor[HTML]{FF0004} 6 \\
            \hline
            Medium     &    \cellcolor[HTML]{00FF1D} 10& \cellcolor[HTML]{FFEA00} & 
                            \cellcolor[HTML]{FFEA00} & \cellcolor[HTML]{FFEA00} 3, 5 &
                            \cellcolor[HTML]{FF0004} 2 \\
            \hline
            Low        &    \cellcolor[HTML]{00FF1D} & \cellcolor[HTML]{00FF1D} 9 & 
                            \cellcolor[HTML]{FFEA00} & \cellcolor[HTML]{FFEA00} 1, 13 & 
                            \cellcolor[HTML]{FFEA00} 4 \\
            \hline
            Very Low   &    \cellcolor[HTML]{00FF1D} & \cellcolor[HTML]{00FF1D} & 
                            \cellcolor[HTML]{00FF1D} 12 & \cellcolor[HTML]{FFEA00} & 
                            \cellcolor[HTML]{FFEA00} 11 \\ [1ex]
            \hline
        \end{tabular}
    \caption{Risk analysis form}
    \label{riskForm}
\end{table}

\subsection{Risk mitigation strategy}

Table \ref{riskMitigation} shows the mitigation strategy for the project risks with highest priority and should give an overview of mitigations that should help with most of the identified risks, as several can be mitigated with similar strategies.

% 4 & Technology & Application instability & 5 & Low & Severe
% 5 & Organization & Team members are unavailable for longer period of time & 6 & Medium & Significant
% 1 & Organization & Losing important data & 7 & Low & Significant
% 10 & Organization & Fail to follow software methodology & 8 & Medium & Minimal
% 9 & Technology & Integration testing environments aren't available & 9 & Low & Minor
% 3 & Technology & Sanitation of input from external sources & 10 & Medium & Significant
% 11 & Technology & Exposure of sensitive personal or corporate data & 11 & Very Low & Severe
% 12 & Technology & Incapability of identifying individual or corporate entity & 12 & Very Low & Moderate
% 13 & Technology & Illegal use of stored software & 13 & Low & Significant
\begin{table}[H]
\resizebox{\columnwidth}{!}{
\begin{tabular}{|c|l|p{10cm}|}
    \hline
    \cellcolor[HTML]{AFAFAF} Priority & \cellcolor[HTML]{AFAFAF} Risk & \cellcolor[HTML]{AFAFAF} Mitigation  \\
    \hline
    \hline
    1 & Discovering problems in requirements at delivery
    & \begin{itemize}
        \item Use an agile development process 
        \item Perform regular user testing
    \end{itemize} \\ 
    \hline
    2 & Learning curve lead to delay 
    & \begin{itemize}
        \item When a new technology is to be used, a workshop for exploring it will be held
    \end{itemize}\\
    \hline
    3 & Project size is under estimate
    & \begin{itemize}
        \item Use an agile development process
        \item Have regular meetings with product owner 
    \end{itemize} \\
    \hline
        4 & Technology components are not scalable
        & \begin{itemize}
            \item Use IaC and code for modularity 
        \end{itemize}\\
    \hline
    5 & Application instability
    &\begin{itemize}
        \item Unit test coverage of all important functions
        \item Use DevOps with live server throughout the development
    \end{itemize} \\
    \hline
    6 & Team members are unavailable for longer period of time
    &\begin{itemize}
        \item Focus on reviewing git pull requests well
        \item Good communication amongst developers
    \end{itemize}\\
    \hline
    7 & Losing important data
    &\begin{itemize}
        \item Use cloud storage
        \item Additional data storage security systems
        \item Have and take regular backups
    \end{itemize}\\
    \hline
    
\end{tabular}
}
\caption{Risk mitigation}
\label{riskMitigation}
\end{table}

\newglossaryentry{git}{name=Git, description={Version control system}}
\newglossaryentry{dosAttack}{name=DOS-attack, description={Denial of Service attack}}
\newglossaryentry{proprietary}{name=proprietary, description={Closed source}}
\newglossaryentry{xssAttack}{name=XSS-attack, description={Cross site scripting, input executable code that will run on other instances of the service}}
\newglossaryentry{sql}{name=SQL, description={Structured Query Language used for Querying form certain database systems}}
\newglossaryentry{sqliAttack}{name=SQLi-attack, description={\gls{sql} injection, input of executable \gls{sql} code that could modify a \gls{sql}-based database system}}

\subsection{Final Security requirements}
\begin{itemize}
    \item Sanitation of input from external sources within visualization and field for repository link submission to mitigate \gls{sqliAttack} and \gls{xssAttack}s
    \item No accounts will mitigate a lot of related responsibilities with regards to security about storing sensitive personal or corporate data .e.g Encryption of sensitive data, authentication of users, secure storage of user data, etc.
    \item The application won't in any way record or process any data that can be used to identify usage patterns, any individual or corporation. 
    \item Application will not store nor expose sensitive personal or corporate project information due to the way \gls{git} handles private or \gls{proprietary} repositories. Any project stored will be freely available on other services such as \href{Github}{https://github.com/}.
    \item The submitted repositories that are stored, are only stored for visualization of the codebase and won't at any point in time be built nor executed by any group member or the application it-self. Any submitted repository will not be used for any other purpose.
    \item Extensive testing to mitigate instabilities.
\end{itemize}

