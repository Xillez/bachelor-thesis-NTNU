%This is for those you will actually be getting code out beyond the end of the thesis.  This describes how the code is deployed on the test servers and into the testing environment.  Having the code running on your own machines is nice, but you need a process so that you can share your code with other people and have it actually run without them having to have a copy of a compiler and recompile your code.

\chapter{Deployment}
\label{chap:deployment}

\section{Dockerization}
\newglossaryentry{mongo}{name={MongoDB}, description={A document database storing data in \gls{json}-like documents \cite{mongodb}}}

\Gls{docker} was used to simplify the system setup for any administrative users and help with documenting the infrastructure. The goal was to be able to launch the system with by running the command: "docker-compose up".

Whit this it would be easier for 
\Gls{dockercompose} names four services:
\begin{itemize}
    \item web - Serves the \gls{html}, \gls{css} and \gls{js}. Mainly deals with the visualization, presenting state and quality metrics.
    \item api - The Go \gls{api} server. Mainly dealing with controlling the database and parser on requests from web. 
    \item mongo\_db - The \gls{mongo} database \cite{mongodb}.
    \item doc - Servs the \gls{apidoc} 
\end{itemize}

The services; web, api and doc each have a \gls{dockerfile} in the repository, while mongo\_db uses mongo:latest, which is the official image. 

\section{Deployment on SkyHigh with HOT}
\newglossaryentry{openstack}{name={openstack}, description={A cloud operating system that controls large pools of compute, storage, and networking resources throughout a datacenter\cite{openstack}}}
\newglossaryentry{skyhigh}{name={SkyHigh}, description={\gls{openstack} instance hosted by NTNU}}
\newacronym{hot}{HOT}{Heat Orchestration Template \cite{openstack:hot}}

As of the writing of this report, the system is deployed at \gls{skyhigh}.
It uses a \gls{hot} to automate the server setup process. 