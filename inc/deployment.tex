%This is for those you will actually be getting code out beyond the end of the thesis.  This describes how the code is deployed on the test servers and into the testing environment.  Having the code running on your own machines is nice, but you need a process so that you can share your code with other people and have it actually run without them having to have a copy of a compiler and recompile your code.

\chapter{Deployment}
\label{chap:deployment}

\section{Dockerization}
\newglossaryentry{mongo}{name={MongoDB}, description={A document database storing data in \gls{json} like documents \cite{mongodb}}}

\Gls{docker} was used to simplify the system setup for any administrative users and help with documenting the infrastructure. The goal was to be able to launch the system by running the command: "docker-compose up".

With this it would be easier for potential administrators to test the system and figure out if it would be appropriate for their use-case. It would also be easier to continue with \gls{opensource} development, as the \gls{docker} instances could be used as a common test bed. 

\Gls{dockercompose} names four services:
\begin{itemize}
    \item web - Serves the \gls{html}, \gls{css} and \gls{js}. Mainly deals with the visualization, presenting state and quality metrics.
    \item api - The Go \gls{api} server. Mainly dealing with controlling the database and parser on requests from web. 
    \item  mongo\_db - The \gls{mongo} database \cite{mongodb}.
    \item doc - Serves the \gls{apidoc} 
\end{itemize}

The services; web, api and doc each have a defined \gls{dockerfile} in the repository, while mongo\_db uses mongo:latest, which is the official image. 

mongo\_db is connected with api but not with the network. This improves the security by not directly exposing the database to the internet.

\section{Deployment on SkyHigh with HOT}
\newglossaryentry{openstack}{name={OpenStack}, description={A cloud operating system that controls large pools of compute, storage, and networking resources throughout a datacenter\cite{openstack}}}
\newglossaryentry{skyhigh}{name={SkyHigh}, description={\gls{openstack} instance hosted by NTNU}}
\newglossaryentry{jenkins}{name={Jenkins}, description={A self-contained, open source automation server which can be used to automate all sorts of tasks related to building, testing, and delivering or deploying software. \cite{jenkins}}}
\newacronym{hot}{HOT}{Heat Orchestration Template \cite{openstack:hot}}

As of the writing of this report, the system is deployed at \gls{skyhigh} and associated with the domain \url{http://www.covi.tech/}.

It uses \gls{hot} to automate the server setup process. The template defines three servers:

\begin{itemize}
    \item development - Meant as a testing server that updates rapidly and might contain bugs.
    \item operations - Meant to be used by the end users and updated with each milestone where most bugs have been patched.
    \item master - Meant to run \gls{jenkins} to automate updates of development and operations.
\end{itemize}

Due to the teams very limited experience with \gls{openstack} and \gls{jenkins}, the servers were rarely updated and down prioritized as it would require more time to learn than what could be allocated. 